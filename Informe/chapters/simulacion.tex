\chapter{Implementación simulada del diseño }
\label{section:simu}


\section{Configuración de la maqueta de red }

Para simular el funcionamiento del diseño planteado anteriormente, se crea en la herramienta PacketTracer un prototipo reducido de la red, imponiendo dos equipos finales por cada subred como se puede observar en la figura \ref{fig:maqueta}.

    \begin{figure}[h]
    	\centering
    	\includegraphics[width=0.65\textwidth]{images/simu/simu}
    	\caption{Maqueta de la red implementada para la simulación en PacketTracer.}
    	\label{fig:maqueta}
    \end{figure}
    
\vspace{2mm}

En cuanto a la capa de enlace, se añaden cuatro switches: tres de ellos conectados a pares de equipos (de grupos de trabajo distintos) y otro más para aportar redundancia y tolerancia a fallos de red. Las interconexiones entre ellos se basan en enlaces en modo trunk que se deben gestionar para evitar posibles bucles mediante la configuración del protocolo \textbf{STP} (ver Figura \ref{fig:stp}). 

\vspace{1mm}

Por ello, se calcula el árbol de red, en el que se impone el switch 0 como raíz y se realizan varios pings para comprobar el correcto funcionamiento y visualizar las rutas seguidas por los mismos. El diseño lógico de la red se obtendrá a partir de los enlaces que se indiquen como designados.

\pagebreak

    \begin{figure}[h]
    \centering
    	\includegraphics[width=0.55\textwidth]{images/simu/stp}
        \includegraphics[width=0.4\textwidth]{images/simu/ruta3}
    	\caption{Configuración STP y ruta seguida por ping entre PC1 y PC0 (grupo 3).}
    	\label{fig:stp}
    \end{figure}
    
\vspace{1mm}
  
En cuanto a la configuración de equipos de capa 3, se establecen dos routers para habilitar la comunicación entre equipos de distintos grupos de trabajo y en su defecto la comunicación con la red pública. 

\vspace{1mm}

Es decir, se plantea un modelo activo-pasivo mediante el protocolo \textbf{HSRP} para proporcionar redundancia a nivel de gateway y por tanto, alta disponibilidad. Se determina el router 0 como activo y el router 1, como standby (ver Figura \ref{fig:standby-direccionamiento}).

\vspace{1mm}

El proceso requiere reservar tres direcciones para cada VLAN. En consecuencia, al configurar \textbf{DHCP}, se excluyen nueve direcciones IP del espacio de direccionamiento, quedando 59 disponibles para cada grupo de trabajo o VLAN como se indica en la tabla~\ref{tab:IP}.

\vspace{3mm}

\begin{table}[h]
\centering
\begin{tabular}{|c|c|c|}
\hline
Dirección IP & VLAN & Descripción \\ \hline
.0  & 1 & Dirección de red \\ \hline
.1  & 1 & vIP de los routers de VLAN 1  \\ \hline
.2  & 1 & Interfaz del router activo (router 0)  \\ \hline
.3  & 1 & Interfaz del router pasivo (router 1) \\ \hline
.4 ... .62 & 1 & Direcciones disponibles  \\ \hline
.63  & 1 & Broadcast VLAN 1 \\ \hline
.64  & 2 & Dirección de red \\ \hline
.65  & 2 & vIP de los routers de VLAN 2 \\ \hline
.66  & 2 & Interfaz del router activo (router 0)  \\ \hline
.67  & 2 & Interfaz del router pasivo (router 1)  \\ \hline
.68 ... .126 & 2 & Direcciones disponibles  \\ \hline
.127  & 2 & Broadcast VLAN 2 \\ \hline
.128  & 3 & Dirección de red \\ \hline
.129  & 3 & vIP de los routers de VLAN 3  \\ \hline
.130  & 3 & Interfaz del router activo (router 0) \\ \hline
.131  & 3 & Interfaz del router pasivo (router 1) \\ \hline
.132 ... .190 & 3 & Direcciones disponibles  \\ \hline
.191  & 3 & Broadcast VLAN 3 \\ \hline

\end{tabular}
\caption{Asignación de direcciones IP para cada subred.}
\label{tab:IP}
\end{table}

\pagebreak

    \begin{figure}[h]
    	\centering
    	\includegraphics[width=0.45\textwidth]{images/simu/standby}
        \includegraphics[width=0.5\textwidth]{images/simu/ip dhcp pool}
    	\caption{Configuración HSRP y direccionamiento asignado para cada VLAN.}
    	\label{fig:standby-direccionamiento}
    \end{figure}
    
\vspace{1mm}

\section{Comprobación del funcionamiento}

Una vez realizadas las configuraciones necesarias en los dispositivos de red, comprobamos que se cumple con los requisitos de conexión en base a la realización de pings (es posible que algunos fallen la primera vez que se realizan):
\begin{itemize}
    \item Conectividad entre los diferentes grupos de trabajo: los pings entre los PC de diferentes VLAN se realizan de forma correcta \vspace{1mm}
    \item Funcionamiento de HSRP: se comprueba que las interfaces virtuales de las tres VLANs funcionan correctamente.
\end{itemize}

\vspace{2mm}

    \begin{figure}[h]
    	\centering
        \includegraphics[width=0.4\textwidth]{images/simu/pings.png}
    	\caption{Pruebas de ping entre los dispositivos.}
    	\label{fig:pings}
    \end{figure}
    

    
    
    



    



