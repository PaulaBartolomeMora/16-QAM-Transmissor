\chapter{Diseño de la red}
\label{section:diseno}


\section{Dimensionado y equipos de red}

\subsection{Cálculo del dimensionado de red}
\label{sec:dim}

Para calcular las dimensiones de la red se debe tener en cuenta como requisito principal que el \textbf{total de puestos es de 90} y que además, se debe imponer un sobre-dimensionamiento de entre un 15\% y 30\%  de la red: 

\vspace{1mm}

Mínimo número de direcciones: $90*1,15 = 103,5 \approx 104$

Máximo número de direcciones: $90*1,3 = 117$ 
 
\vspace{1mm}

En cuanto a la instalación de switches, observamos que se nos presentan varias opciones posibles de configuración. Por un lado, podemos realizarla de forma homogénea, imponiendo todos los switches del mismo número de puertos:

\vspace{2mm}
 
\begin{itemize}
\item Switches de 8 puertos: $104/8 = 13$ switches $\rightarrow 13*8 = 104$ puertos disponibles (15\% de sobredimensionado). Sería una opción válida en términos de sobredimensionado, pero al requerirse un número elevado de switches supone un aumento de los costes del diseño de red. \vspace{1mm}
\item Switches de 24 puertos: $104/24 = 4.33$ switches $\rightarrow 5*24 = 120$ puertos disponibles (33\% de sobredimensionado). Opción no válida, se supera el sobredimensionado recomendado. \vspace{1mm}
\item Switches de 48 puertos: $104/48 = 2.16$ switches $\rightarrow 3*48 = 144$ puertos disponibles (60\% de sobredimensionado). Opción no válida, se supera el sobredimensionado recomendado. \vspace{1mm}
\end{itemize}

\vspace{1mm}

Como ninguna de las posibilidades anteriores es adecuada, se decide buscar una solución heterogénea, haciendo uso de switches de diferentes características: 

\begin{itemize}
\item 48+48+8: se utiliza el menor número de switches cumpliendo con los requisitos de dimensionamiento. \vspace{2mm} \pagebreak
\item 48+24+24+8: en esta opción se utiliza un switch más que en el caso anterior. Realizando comparaciones entre distintos modelos de 24 y 48 puertos observamos que en términos de costes no supone una opción mejor \cite{cisco_switches}\cite{tplink_switches}\cite{huawei_switches}.\vspace{2mm}
\end{itemize}

Por lo tanto, con todo lo especificado anteriormente, se decide que la versión de configuración de red más eficiente es la que añade 3 switches (48+48+8), obteniéndose \textbf{104 puertos en total}. No obstante, como se debe aportar redundancia y tolerancia ante fallos se añade un switch más de 8 puertos que no proporciona acceso a los clientes directamente.

\vspace{1mm}

\subsection{Selección de modelos de equipos de red}

\vspace{1mm}

\begin{itemize}
\item Routers Catalyst 8300-1N1S-6T \cite{cat8300}: con 6 puertos Gigabit Ethernet. \vspace{1mm}
\item Switches de cisco de la serie Catalyst 9200 C9200-48PGX y C9200CX-8P SKUs \cite{cat9200}.
\end{itemize}

\vspace{2mm}

    \begin{figure}[h]
        \begin{subfigure}[b]{\textwidth}
            \centering
            \includegraphics[width=0.65\textwidth]{images/diseno/sw_nvlans.png}
        \end{subfigure}
        \begin{subfigure}[b]{\textwidth}
            \centering
            \includegraphics[width=0.65\textwidth]{images/diseno/sw_model_48.png}
            
        \end{subfigure}
        \begin{subfigure}[b]{\textwidth}
            \centering
            \includegraphics[width=0.65\textwidth]{images/diseno/sw_model_8.png} 
        \end{subfigure}
    	\caption{Especificaciones modelos de switch.}
    	\label{fig:specs}
    \end{figure}


\section{Selección de cableado}

En este apartado llevamos a cabo la comparación de las características de las diferentes categorías de cableado a elegir \cite{cableado1} \cite{cableado2}. Como en nuestro caso disponemos de puertos de 1Gb debemos tener en cuenta opciones de cableado que se encuentren por encima de la categoría 5 (a partir de la categoría 5E):

\vspace{2mm}

\begin{table}[h]
\centering
\begin{tabular}{|c|c|c|}
\hline
Categoría & Max data rate & Bandwidth \\ \hline
5e & 1Gbps & 100MHz \\ \hline
6 & 1Gbps & 250MHz \\ \hline
6a & 10Gbps & 500MHz \\ \hline
\end{tabular}
\caption{Características de cada categoría de cableado.}
\end{table}

\vspace{2mm}

En cuanto a la Categoría 5E, observamos que se ajustaría más a un diseño residencial. Se podría quedar corto para nuestra implementación al suponer un ancho de banda de 100MHz, por lo que la descartamos como opción.

\vspace{1mm}

Por consiguiente, valorando entre la categoría 6 y 6A, tomamos la decisión de seleccionar la primera porque su velocidad es de 1Gbps y tiene un menor coste que la segunda. Sería erróneo hacer uso de cableado de categoría 6A, ya que tendría como consecuencia un sobre-dimensionamiento de la red al aportar una velocidad de 10Gbps. 

\vspace{1mm}

Respecto al apantallamiento, se decide emplear cables FTP (apantallamiento global) para minimizar los posibles efectos de las interferencias al contar con mazos agregados de cables.
    
\section{Diseño del esquema lógico de red}

A cada colectivo se le asigna una VLAN diferente para dividir el tráfico, por lo que se decide dividir el rango de direcciones asignado (172.19.7.0/24) de la siguiente forma:
\begin{itemize}
    \item VLAN1 172.19.7.0/26 (.7.0 hasta .7.63)
    \item VLAN2 172.19.7.64/26 (.7.64 hasta .7.127)
    \item VLAN3 172.19.7.128/26 (.7.128 hasta .7.191)
\end{itemize}


\section{Distribución de equipos de trabajo}

El diseño planteado debe desplegarse en las salas 1, 2 y 4 del espacio de coworking (figura \ref{fig:salas}). Debido a las diferencias en el espacio disponible las salas 1 y 4 albergan aproximadamente la mitad de puestos que la sala 2 que cuenta con 44.

\vspace{2mm}

    \begin{figure}[h]
    	\centering
    	\includegraphics[width=0.65\textwidth]{images/diseno/salas_enunciado.png}
    	\caption{Distribución de las salas sobre las que desplegar la red.}
    	\label{fig:salas}
    \end{figure}
    
\vspace{2mm}

Utilizando el cableado seleccionado en el apartado anterior se realiza el despliegue a través del suelo técnico para llegar a todos los puestos (ver Figura \ref{fig:puestos_cables}). Como se comentaba en el Apartado \ref{sec:dim}, podemos llegar a disponer de 104 puestos debido al sobredimensionamiento. 

En consecuencia, se obtiene un total de 208 cables, consiguiendo dos bocas por puesto y proporcionando conexiones entre estas y los patch panels. Además, se calcula que el cable de mayor longitud tendría alrededor de 30 metros.

\vspace{3mm}

    \begin{figure}[h]
    	\centering
    	\includegraphics[width=0.5\textwidth]{images/diseno/diagrama_puestos_cableado.drawio.png}
    	\caption{Distribución de los puestos y cableado en las salas.}
    	\label{fig:puestos_cables}
    \end{figure}
    

\section{Planificación del rack y asignación de puertos}

Para exponer la asignación de puertos se toma como ejemplo la configuración realizada para el switch 0 en la tabla \ref{tab:puertos}.

\vspace{2mm}

\begin{table}[h]
\centering
    \begin{tabular}{|c|c|c|c|}
    \hline
    \textbf{Physical} & \textbf{VLAN/Trunk} & \textbf{Device} & \textbf{Remote interface} \\ \hline
    G0/1 & Trunk & Switch1 & G0/2 \\ \hline
    G0/2 & Trunk & Switch2 & G0/1 \\ \hline
    G0/3 & Trunk & Switch3 & G0/3 \\ \hline
    G0/4 & Trunk & Router0 & G0/0 \\ \hline
    G0/5 & 1 & PC0 - group 1 & G0 \\ \hline
    G0/6 & 2 & PC0 - group 2 & G0 \\ \hline
    \end{tabular}
    \caption{Ejemplo de asignación de puertos del switch 0.}
     \label{tab:puertos}
\end{table}

En cuanto al diseño del rack (ver Figura \ref{fig:rack}), se incorporan al mismo los 4 switches necesarios y 2 routers para posibilitar la comunicación entre equipos de distintos grupos de trabajo con redundancia. Además, cabe destacar que se incorporan a la red patch panels modulares para controlar y gestionar el cableado de una forma más eficiente.

\vspace{3mm}

\begin{table}[h]
    \centering
    \begin{tabular}{|c|c|c|c|}
    \hline
    \textbf{Name} & \textbf{Device} & \textbf{Function} & \textbf{Location} \\ \hline
    Core1 & 8300-1N1S-6T & Core router & Rack \#1 \\ \hline
    Core2 & 8300-1N1S-6T & Core router & Rack \#1 \\ \hline
    Switch0 & C9200-48PGX & Access sw & Rack \#1 \\ \hline
    Switch1 & C9200CX-8P & Access sw & Rack \#1 \\ \hline
    Switch2 & C9200-48PGX & Access sw & Rack \#1 \\ \hline
    Switch3 & C9200CX-8P & Redundancy sw & Rack \#1 \\ \hline
    \end{tabular}
    \caption{Definición y ubicación de equipos.}
\end{table}

    \begin{figure}[h]
    	\centering
    	\includegraphics[width=0.3\textwidth]{images/diseno/rack.png}
    	\caption{Planificación del rack}
    	\label{fig:rack}
    \end{figure}

\pagebreak



