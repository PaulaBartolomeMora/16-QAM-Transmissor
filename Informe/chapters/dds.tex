\chapter{Mezclador de Transmisión}
\label{section:mezcla}

En este Capítulo se describe el bloque de mezclador de transmisión.

generación de componenes


encargado de la modulación 16-QAM. Este bloque se compone de tres procesos: mapeado 16-QAM, aplicación de Zero Padding y filtrado \textit{Root Raised Cosine}. En la Figura \ref{fig:qam_fir} se puede visualizar la estructura de los bloques IP que se han incluido en este segundo proyecto de Vivado. 

%Para generar las componentes I/Q de transmisión se multiplican los datos I/Q generados anteriormente por los valores de las funciones coseno/-seno obtenidas por un sistema DDS. Las muestras del DDS y Salidas_RRC se procesan y multiplicand a 3x192 MHz, esto es 576 MHz Se generan señales sinusoidales coseno/-seno de frecuencia 100 MHz (5 / 6 muestras por ciclo) Para la multiplicación se hará uso de un bloque IP multiplicador. El número de bits de cada bloque es una consideración global del diseño realizado.


\section{Generación de las sinusoides coseno/-seno mediante bloque DDS a 576 MHz}



\section{Bloque de multiplicación y suma de señales de transmisión I/Q. Comprobación de datos}



\section{Configuración y uso del bloque MMCM en el diseño}






