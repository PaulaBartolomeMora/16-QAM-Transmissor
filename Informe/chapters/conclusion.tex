\chapter{Conclusiones}
\label{section:conclusiones}

Con este trabajo se ha conseguido aportar de forma significativa un mayor entendimiento sobre los sistemas de comunicaciones avanzados, especialmente sobre los basados en 16-QAM. Mediante la aplicación de los conceptos estudiados teóricamente en la asignatura y de las tecnologías explicadas en el laboratorio, se ha conseguido implementar todos los módulos del sistema propuesto, logrando el objetivo principal del proyecto.

\vspace{3mm}

Para ello, ha sido necesario un análisis en profundidad de la documentación dada por Xilinx en referencia con el funcionamiento de cada uno de los bloques a desarrollar. Este paso ha sido de gran importancia para garantizar una correcta configuración de los mismos. 

\vspace{3mm}

Por otro lado, la validación del funcionamiento de los módulos se ha conseguido a partir del análisis cuantitativo y cualitativo de los resultados obtenidos en las simulaciones funcionales. Por ello, se han diseñado ficheros de testbench para cada módulo implementado. La continua evaluación del código ha permitido en ciertos momentos el reajuste de las configuraciones realizadas con el fin de corregir errores y optimizar el sistema.

\vspace{3mm}

Finalmente, se puede expresar que se ha logrado una integración completa  de todos los módulos de forma exitosa mediante el desarrollo de procesos de lectura y escritura de ficheros de entrada y salida en el testbench para conseguir la cohesión en todo el sistema.

